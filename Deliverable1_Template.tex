\documentclass[]{article}

% Imported Packages
%------------------------------------------------------------------------------
\usepackage{amssymb}
\usepackage{amstext}
\usepackage{amsthm}
\usepackage{amsmath}
\usepackage{enumerate}
\usepackage{fancyhdr}
\usepackage[margin=1in]{geometry}
\usepackage{graphicx}
\usepackage{extarrows}
\usepackage{setspace}
%------------------------------------------------------------------------------

% Header and Footer
%------------------------------------------------------------------------------
\pagestyle{plain}  
\renewcommand\headrulewidth{0.4pt}                                      
\renewcommand\footrulewidth{0.4pt}                                    
%------------------------------------------------------------------------------

% Title Details
%------------------------------------------------------------------------------
\title{Deliverable \#1 Template}
\author{SE 3A04: Software Design II -- Large System Design}
\date{}                               
%------------------------------------------------------------------------------

% Document
%------------------------------------------------------------------------------
\begin{document}

\maketitle	

\section{Introduction}
\label{sec:introduction}
% Begin Section
\subsection{Purpose}

The purpose of this SRS is to describe the intended purpose and environment of our system WhoDatDog for future developers and software design teams
]. 


\subsection{Scope}
\label{sub:scope}
% Begin SubSection
The system application shall be called WhoDatDog, The software shall attempt to identify a dog through a description fed from the phone owners. As an additional feature, after the system has identified the dog it should allow the user to share the identified dog on twitter. Another additional feature is that the user should be given a list of the nearest pet stores for the type of dog identified. The benefits of this system is to be able to conveniently identify any type of dog with your phone. The objective of this system is to allow the user to identity a dog through a description on their mobile phone. Our goal is to create an simple and effective android application that can identify a dog. The relevant benefits of this system is the user now can identify a dog through automation, which is faster and more convenient then the traditional methods.   

\subsection{Definitions, Acronyms, and Abbreviations}
\label{sub:definitions_acronyms_and_abbreviations}
% Begin SubSection
\begin{enumerate}[a)]
	\item \textbf{Android Device} - A device such as a smart phone or tablet that runs on an Android operating system.
\end{enumerate}

\subsection{Acronyms and Abbreviations}
\begin{enumerate}[a)]
	\item \textbf{WWD} - an acronym for WhoDatDog
	\item \textbf{API} - an acronym for application program interface
\end{enumerate}

% End SubSection

\subsection{References}
N/A

\subsection{Overview}
\label{sub:overview}
% Begin SubSection
The rest of the document contains the overall description, functional requirements, and non functional requirements. The overall description of the system should describe the general factors that affect the WhoDatDog system and its requirements. The functional requirements section should contain the details for a designer to design the system. It is organized by first business events and then by viewpoints. The non functional requirements should contain the qualities our system wants. The SRS is organized sequentially from the sections mentioned above. 
\section{Overall Description}
\label{sec:overall_description}
% Begin Section

This section of the SRS should describe the general factors that affect the product and its requirements. It does not state specific requirements; it provides a background for those requirements and makes them easier to understand.

\subsection{Product Perspective}
\label{sub:product_perspective}
% Begin SubSection
The Dog Identifier is an application that identifies the breed of a dog. Much like the application Dogsnap in the Appstore but instead of photo recognition the system will use a description of the dog input by the user.\\

The Dog Identifier in this SRS is an application that will be used on a larger Android system. The software therefore must be able to interact with the Android system's physical mechanisms.\\

The application will have an interface that can be acessed by the user on an Android device. The figure 1 below shows how the application will interact with the larger system.
% End SubSection

\subsection{Product Functions}
\label{sub:product_functions}
% Begin SubSection
This subsection will describe the main functions the Dog Identifier must have.

\begin{itemize}
	\item \textbf{Identify Dog:} the software  shall give the user a list of all the dogs that match the characteristics inputted
	\item \textbf{Social Media Post:} Using Google Maps API, the system shall make a post to the users social media platform about the dog that was identified and where the user was at the time
	\item \textbf{Pet Store Locator:} Using Google Maps API, the system shall give the user a list of pet stores arranged by closest distance from the user
\end{itemize}
% End SubSection

\subsection{User Characteristics}
\label{sub:user_characteristics}
% Begin SubSection
The User must:
\begin{itemize}
	\item Understand how to use the interface of the Android device to access the Dog Identifier
	\item Be able to read/understand English to interact with the interface
	\item Be able to input characteristics of a dog
\end{itemize}
% End SubSection

\subsection{Constraints}
\label{sub:constraints}
% Begin SubSection
Time constraints are the main general constraints of the Dog Identifier system. There will be project dates for specific deliverables that the team must meet. The functional and nonfunctional requirements will also constrain all the features of the software.
% End SubSection

\subsection{Assumptions and Dependencies}
\label{sub:assumptions_and_dependencies}
% Begin SubSection
We are assuming that the user will have a working Android device that has the Dog Identifier application installed. We are assuming that the Android device has access and is connected to the internet. Finally, we are assuming that the application and device are both running.
% End SubSection

\subsection{Apportioning of Requirements}
\label{sub:apportioning_of_requirements}
% Begin SubSection

% End SubSection

% End Section

\section{Functional Requirements}
\label{sec:functional_requirements}
% Begin Section
The following functional requirements are defined firstly by business event (BE1, BE2, ...), then by viewpoint (VP1, VP2, ...). The viewpoints identified in these functional requirements are defined as followed:
\begin{itemize}
\item User (The person(s) who use the software program. These individuals may or may not have dogs of their own)
\item Dog owners (Individuals who own dogs of their own, but who do not use the software themmselves. It is expected that these owners' dogs are the ones which the Users would be describing in the software)
\item Software developers 
\item Business owners (Individuals who own/manage pet stores, kennels, dog pounds, etc. businesses where a User may purchase or adopt a dog)
\end{itemize}

\begin{enumerate}[{BE}1.]
	\item The user sees a dog and requests a system search
	\begin{enumerate}[{VP1}.1]
		\item Users
			\begin{enumerate}
				\item The system must be able to recognize and identify user input, by either reading user input or allowing the user to choose from several preset inputs.
				\item The system must contain or hold a database of dog breeds. The system must be able to search throughthis database via inquiries. Being able to add/remove elements to/from the database is not important and is not the purpose of this system.
				\item The system must contain at least 3 “expert” subsystems. These experts’ purposes are to simultaneously scan the user’s input and find keywords related to their areas of expertise to then search in the database.
                \begin{enumerate}[{iii}.i]
                \item Each expert must be able to react and identify particular keywords in the user’s input. For example, if one expert is an expert on fur color/pattern, they will search the input for words related to colors or patterns.
                \item Each expert’s specialty must be distinctive enough such that one keyword can only be extracted and used by one expert.
				\item If an expert is unable to find any related keyword, it will return an error message, but this will not stall/halt the execution of the other experts.
				\item Once all experts have found keywords in the input, they will create a query by which to search the dog breed database.
				\item The result of the database query will be returned to the user.
                \end{enumerate}
                \item The system will use a “forum”, similar to a social networking system, in which the user will ask questions and to which the system will respond.
				\item The system will respond to user’s text inputs with a text output of its’ own.
			\end{enumerate}
		\item Dog Owners
			\begin{enumerate}
				\item The system will not require additional input from a dog owner other than the system’s current user.
			\end{enumerate}
     	\item Software Developers
        	\begin{enumerate}
        	\item The system must perform exactly as specified in BE1. VP 1.1.
			\item The system will only require access to the developer’s servers in the case that a system update must be performed.
			\item The system will not require input from a developer in order for its’ identification system to be used.
            \end{enumerate}
      	\item Business Owners\\(Not applicable)
	\end{enumerate}
\item The program uses the user location and, if available, a query from 			the dog breed database, to search for pet stores, kennels, dog 				pounds, etc. related business locations in the user’s vicinity.
	\begin{enumerate}[{VP2}.1]
		\item Users
			\begin{enumerate}
				\item The system will be able to use Google Maps integration to find the user’s geographic location, for the purpose of locating nearby businesses where the user may purchase a dog, within some specified distance from the user.
				\item At the user’s request, the system can use the result of the database search conducted in BE1.VP1.1.iii.iv as an additional search parameter. More specifically, the system can be set to search for only businesses which sell that particular breed of dog found with the query.
			\end{enumerate}
		\item Dog Owners\\(Not applicable)
		\item Software Developers
        	\begin{enumerate}
            \item The software must not intrude into users private information, and will not require personal information regardless.
            \end{enumerate}
       	\item Business Owners
        	\begin{enumerate}
            \item The system must be able to display nearby businesses to the user where the user may purchase/adopt a dog.
			\item The system must be able to access information regarding the address of the businesses in question.
			\item The system must be able to access online listings of dog for sale/adoption at the business, and be able to search through these listings for breeds matching the result of the query in BE1.VP1.1.iii.iv.
			\item The system will not have permission to access any other information about the business, aside from its’ name, products for sale, and information required for points BE2.VP2.4.ii and BE2.VP2.4.iii, above.
            \end{enumerate}
	\end{enumerate}
    \item Once the user has identified a dog using BE1, the user can create a social media post describing the dog they have identified and their current geographical location.
    	\begin{enumerate}[{VP3}.1]
        \item Users
        	\begin{enumerate}
            \item The system will be able to use Google Maps API to identify the user's current location.
            \item The system will be able to connect to a number of social media services (etc. Facebook, Twitter, etc.) in order to make a post on the user's behalf.
            	\begin{enumerate}[{ii}.i]
            	\item The system will require the user's permission in order to make a social media post.
            	\item The system will be able to include textual information regarding the last dog breed identified by the user in the social media post.
            	\item The system will be able to attach a photograph to the social media post, if the user gives permission. The system will be able to use either the user's device's camera or access the photographs stored on the user's device.
                \item The user will be able to specify if their social media post is one describing a lost dog.
            	\end{enumerate}
            \end{enumerate}
     	\item Dog Owners
        	\begin{enumerate}
            \item Dog Owners who use socialmedia sites will be able to view other system users' social media posts regarding lost dogs.
            \end{enumerate}
        \item Software Developers
        	\begin{enumerate}
            \item The system will not require additional permissions from the developers to work.
            \end{enumerate}
      	\item Business Owners\\(Not applicable)
        \end{enumerate}
\end{enumerate}



% End Section

\section{Non-Functional Requirements}
\label{sec:non-functional_requirements}
% Begin Section
\subsection{Look and Feel Requirements}
\label{sub:look_and_feel_requirements}
% Begin SubSection

\subsubsection{Appearance Requirements}
\label{ssub:appearance_requirements}
% Begin SubSubSection
\begin{enumerate}[{LF}1. ]
	\item The product shall have a simple and intuitive UI.
\end{enumerate}
% End SubSubSection

\subsubsection{Style Requirements}
\label{ssub:style_requirements}
% Begin SubSubSection
\begin{enumerate}[{LF}2. ]
	\item The product shall adjust its appearance depending on the system its operating in.
\end{enumerate}
% End SubSubSection

% End SubSection

\subsection{Usability and Humanity Requirements}
\label{sub:usability_and_humanity_requirements}
% Begin SubSection

\subsubsection{Ease of Use Requirements}
\label{ssub:ease_of_use_requirements}
% Begin SubSubSection
\begin{enumerate}[{UH}1. ]
	\item The product shall be easy for 11-year-old children to use.
\end{enumerate}
% End SubSubSection

\subsubsection{Personalization and Internationalization Requirements}
\label{ssub:personalization_and_internationalization_requirements}
N/A

\subsubsection{Learning Requirements}
\label{ssub:learning_requirements}
% Begin SubSubSection
\begin{enumerate}[{UH}2. ]
	\item The product shall be easy for a smartphone user to learn.
\end{enumerate}
% End SubSubSection

\subsubsection{Understandability and Politeness Requirements}
\label{ssub:understandability_and_politeness_requirements}
% Begin SubSubSection
\begin{enumerate}[{UH}3. ]
	\item The product shall have straight forward and unambiguous buttons.
	\item[{UH}4. ] The product shall not use any offensive languages.
\end{enumerate}
% End SubSubSection

\subsubsection{Accessibility Requirements}
\label{ssub:accessibility_requirements}
% Begin SubSubSection
\begin{enumerate}[{UH}5. ]
	\item The product shall be accessible despite the environment, however, a question can only be answered when internet access is granted.
\end{enumerate}
% End SubSubSection

% End SubSection

\subsection{Performance Requirements}
\label{sub:performance_requirements}
% Begin SubSection

\subsubsection{Speed and Latency Requirements}
\label{ssub:speed_and_latency_requirements}
% Begin SubSubSection
\begin{enumerate}[{PR}1. ]
	\item Any interface between a user and the automated product must have a maximum response time of 2 seconds.
\end{enumerate}
% End SubSubSection

\subsubsection{Safety-Critical Requirements}
\label{ssub:safety_critical_requirements}
N/A

\subsubsection{Precision or Accuracy Requirements}
\label{ssub:precision_or_accuracy_requirements}
% Begin SubSubSection
\begin{enumerate}[{PR}2. ]
	\item All numeric amounts must be accurate to 1 decimal place.
\end{enumerate}
% End SubSubSection

\subsubsection{Reliability and Availability Requirements}
\label{ssub:reliability_and_availability_requirements}
% Begin SubSubSection
\begin{enumerate}[{PR}3. ]
	\item The product shall be available for use 24 hours a day, 365 days a year unless under maintenance.
\end{enumerate}
% End SubSubSection

\subsubsection{Robustness or Fault-Tolerance Requirements}
\label{ssub:robustness_or_fault_tolerance_requirements}
% Begin SubSubSection
\begin{enumerate}[{PR}4. ]
	\item The product shall work when it receives information in various formats as long as there is no misspellings.
	\item[{PR}5. ] The product shall keep and display the information user entered next time it's opened unless deleted by the user.
\end{enumerate}
% End SubSubSection

\subsubsection{Capacity Requirements}
\label{ssub:capacity_requirements}
% Begin SubSubSection
\begin{enumerate}[{PR}6. ]
	\item The product shall keep up to 10 of the most recently asked questions.
\end{enumerate}
% End SubSubSection

\subsubsection{Scalability or Extensibility Requirements}
\label{ssub:scalability_or_extensibility_requirements}
% Begin SubSubSection
\begin{enumerate}[{PR}7. ]
	\item The product shall be able to have a greater capacity as well as more features when more time is invested in future developments.
\end{enumerate}
% End SubSubSection

\subsubsection{Longevity Requirements}
\label{ssub:longevity_requirements}
% Begin SubSubSection
\begin{enumerate}[{PR}8. ]
	\item The product shall be available to use forever.
\end{enumerate}
% End SubSubSection

% End SubSection

\subsection{Operational and Environmental Requirements}
\label{sub:operational_and_environmental_requirements}
% Begin SubSection

\subsubsection{Expected Physical Environment}
\label{ssub:expected_physical_environment}
N/A

\subsubsection{Requirements for Interfacing with Adjacent Systems}
\label{ssub:requirements_for_interfacing_with_adjacent_systems}
% Begin SubSubSection
\begin{enumerate}[{OE}1. ]
	\item The product shall not interfere with other apps running on the same device as well as the enrvironment.
\end{enumerate}
% End SubSubSection

\subsubsection{Productization Requirements}
\label{ssub:productization_requirements}
N/A

\subsubsection{Release Requirements}
\label{ssub:release_requirements}
N/A

% End SubSection

\subsection{Maintainability and Support Requirements}
\label{sub:maintainability_and_support_requirements}
% Begin SubSection

\subsubsection{Maintenance Requirements}
\label{ssub:maintenance_requirements}
% Begin SubSubSection
\begin{enumerate}[{MS}1. ]
	\item The product shall be under maintenance at the end of each month.
\end{enumerate}
% End SubSubSection

\subsubsection{Supportability Requirements}
\label{ssub:supportability_requirements}
% Begin SubSubSection
\begin{enumerate}[{MS}2. ]
	\item The product shall be able to work with other apps installed on the device without the need for additional programming.
\end{enumerate}
% End SubSubSection

\subsubsection{Adaptability Requirements}
\label{ssub:adaptability_requirements}
% Begin SubSubSection
\begin{enumerate}[{MS}3. ]
	\item The product shall be able to work in any android based device without the need for additional programming.
\end{enumerate}
% End SubSubSection

% End SubSection

\subsection{Security Requirements}
\label{sub:security_requirements}
% Begin SubSection

\subsubsection{Access Requirements}
\label{ssub:access_requirements}
% Begin SubSubSection
\begin{enumerate}[{SR}1. ]
	\item The product shall only grant access to authorized users.
\end{enumerate}
% End SubSubSection

\subsubsection{Integrity Requirements}
\label{ssub:integrity_requirements}
% Begin SubSubSection
\begin{enumerate}[{SR}2. ]
	\item The product shall not allow unauthorized modification of user information.
	\item[{SR}3. ] The product must encrypt all transmitted messages.
\end{enumerate}
% End SubSubSection

\subsubsection{Privacy Requirements}
\label{ssub:privacy_requirements}
% Begin SubSubSection
\begin{enumerate}[{SR}3. ]
	\item The product shall not allow any unauthorized apps or users to access any information it holds.
\end{enumerate}
% End SubSubSection

\subsubsection{Audit Requirements}
\label{ssub:audit_requirements}
N/A

\subsubsection{Immunity Requirements}
\label{ssub:immunity_requirements}
% Begin SubSubSection
\begin{enumerate}[{SR}4. ]
	\item The product shall place top priority on the protection of user information from infection by unauthorized malicious programs.
\end{enumerate}
% End SubSubSection

% End SubSection

\subsection{Cultural and Political Requirements}
\label{sub:cultural_and_political_requirements}
% Begin SubSection

\subsubsection{Cultural Requirements}
\label{ssub:cultural_requirements}
% Begin SubSubSection
\begin{enumerate}[{CP}1. ]
	\item The product shall not offend any user because of its name, colour or UI.
\end{enumerate}
% End SubSubSection

\subsubsection{Political Requirements}
\label{ssub:political_requirements}
% Begin SubSubSection
\begin{enumerate}[{CP}2. ]
	\item The product shall not use or infer any political figures or icons in its use.
\end{enumerate}
% End SubSubSection

% End SubSection

\subsection{Legal Requirements}
\label{sub:legal_requirements}
% Begin SubSection

\subsubsection{Compliance Requirements}
\label{ssub:compliance_requirements}
N/A

\subsubsection{Standards Requirements}
\label{ssub:standards_requirements}
% Begin SubSubSection
\begin{enumerate}[{LR}1. ]
	\item Personal information must be implemented such that it complies with the Date Protection Act.
\end{enumerate}
% End SubSubSection

% End SubSection

% End Section

\appendix
\section{Division of Labour}
\label{sec:division_of_labour}
% Begin Section
Include a Division of Labour sheet which indicates the contributions of each team member. This sheet must be signed by all team members.
% End Section

\newpage
\section*{IMPORTANT NOTES}
\begin{itemize}
	\item Be sure to include all sections of the template in your document regardless whether you have something to write for each or not
	\begin{itemize}
		\item If you do not have anything to write in a section, indicate this by the \emph{N/A}, \emph{void}, \emph{none}, etc.
	\end{itemize}
	\item Uniquely number each of your requirements for easy identification and cross-referencing
	\item Highlight terms that are defined in Section~1.3 (\textbf{Definitions, Acronyms, and Abbreviations}) with \textbf{bold}, \emph{italic} or \underline{underline}
	\item For Deliverable 1, please highlight, in some fashion, all (you may have more than one) creative and innovative features. Your creative and innovative features will generally be described in Section~2.2 (\textbf{Product Functions}), but it will depend on the type of creative or innovative features you are including.
\end{itemize}


\end{document}
%------------------------------------------------------------------------------